
\documentclass[conference]{IEEEtran}
% \documentclass[conference]{IEEEtran
\IEEEoverridecommandlockouts
% The preceding line is only needed to identify funding in the first footnote. If that is unneeded, please comment it out.
\usepackage{cite}
\usepackage[margin=1in]{geometry}
\usepackage{amsmath,amsthm,amssymb}
\usepackage{algorithmic}
\usepackage{graphicx} 
\graphicspath{ {./pics/} }
\usepackage{subfig}
\usepackage{wrapfig}
\def\BibTeX{{\rm B\kern-.05em{\sc i\kern-.025em b}\kern-.08em
    T\kern-.1667em\lower.7ex\hbox{E}\kern-.125emX}}

% This is to include code
\usepackage{listings}
\usepackage{xcolor}
\definecolor{dkgreen}{rgb}{0,0.6,0}
\definecolor{gray}{rgb}{0.5,0.5,0.5}
\definecolor{mauve}{rgb}{0.58,0,0.82}
\lstdefinestyle{Python}{
    language        = Python,
    basicstyle      = \ttfamily,
    keywordstyle    = \color{blue},
    keywordstyle    = [2] \color{teal},
    stringstyle     = \color{green},
    commentstyle    = \color{red}\ttfamily
}

\newcommand{\N}{\mathbb{N}}
\newcommand{\Z}{\mathbb{Z}}

\newenvironment{theorem}[2][Theorem]{\begin{trivlist}
\item[\hskip \labelsep {\bfseries #1}\hskip \labelsep {\bfseries #2.}]}{\end{trivlist}}
\newenvironment{lemma}[2][Lemma]{\begin{trivlist}
\item[\hskip \labelsep {\bfseries #1}\hskip \labelsep {\bfseries #2.}]}{\end{trivlist}}
\newenvironment{exercise}[2][Exercise]{\begin{trivlist}
\item[\hskip \labelsep {\bfseries #1}\hskip \labelsep {\bfseries #2.}]}{\end{trivlist}}
\newenvironment{reflection}[2][Reflection]{\begin{trivlist}
\item[\hskip \labelsep {\bfseries #1}\hskip \labelsep {\bfseries #2.}]}{\end{trivlist}}
\newenvironment{proposition}[2][Proposition]{\begin{trivlist}
\item[\hskip \labelsep {\bfseries #1}\hskip \labelsep {\bfseries #2.}]}{\end{trivlist}}
\newenvironment{corollary}[2][Corollary]{\begin{trivlist}
\item[\hskip \labelsep {\bfseries #1}\hskip \labelsep {\bfseries #2.}]}{\end{trivlist}}

\begin{document}

% --------------------------------------------------------------
%                         Start here
% --------------------------------------------------------------

\title{MIALab Project HS2020}

\author{\IEEEauthorblockN{1\textsuperscript{st} Thomas Buchegger}
\IEEEauthorblockA{\textit{University of Bern} \\
Bern, Switzerland \\
thomas.buchegger@students.unibe.ch}
\and
\IEEEauthorblockN{2\textsuperscript{nd} Carolina Duran}
\IEEEauthorblockA{\textit{University of Bern} \\
Bern, Switzerland \\
carolina.duran@students.unibe.ch}
\and
\IEEEauthorblockN{3\textsuperscript{rd} Stefan Weber}
\IEEEauthorblockA{\textit{University of Bern} \\
Bern, Switzerland \\
stefan.weber@students.unibe.ch}
}
\date{03.06.2020}

\maketitle



% --------------------------------------------------------------
% Abstract
% --------------------------------------------------------------

\newpage

\begin{abstract}
Here is the abstract


\end{abstract}

\newpage

% --------------------------------------------------------------
% Introduction
% --------------------------------------------------------------

\section{Introduction}
Here is the introduction

Our hypothesis: "Image normalization has an important influence on the segmentation".

Some questions which could be answered:
Currently, z score normalization is implemented.
\begin{itemize}
\item Are there more powerful normalization methods?
\item Is normalization really needed?
\item Is the provided data already normalized in some way?
\item Can you also unnormalize data to show a negative effect?
\end{itemize}

\section{Methodology}

\subsection{Data}
\subsection{Evaluation}

\section{Results}
\section{Discussion}

\section*{Acknowledgment}

\begin{thebibliography}{00}
    \bibitem{b1} Test reference 1
    \bibitem{b2} Test reference 2
\end{thebibliography}
\end{document}
