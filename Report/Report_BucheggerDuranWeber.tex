
\documentclass[journal]{IEEEtran}
% \documentclass[conference]{IEEEtran
\IEEEoverridecommandlockouts
% The preceding line is only needed to identify funding in the first footnote. If that is unneeded, please comment it out.
\usepackage{cite}
\usepackage[margin=1in]{geometry}
\usepackage{amsmath,amsthm,amssymb}
\usepackage{algorithmic}
\usepackage{graphicx} 
\graphicspath{{./figures/}{../png/} }
\usepackage{subfig}
\usepackage{wrapfig}
\def\BibTeX{{\rm B\kern-.05em{\sc i\kern-.025em b}\kern-.08em
    T\kern-.1667em\lower.7ex\hbox{E}\kern-.125emX}}

% This is to include code
\usepackage{listings}
\usepackage{xcolor}
\definecolor{dkgreen}{rgb}{0,0.6,0}
\definecolor{gray}{rgb}{0.5,0.5,0.5}
\definecolor{mauve}{rgb}{0.58,0,0.82}
\lstdefinestyle{Python}{
    language        = Python,
    basicstyle      = \ttfamily,
    keywordstyle    = \color{blue},
    keywordstyle    = [2] \color{teal},
    stringstyle     = \color{green},
    commentstyle    = \color{red}\ttfamily}

\newcommand{\N}{\mathbb{N}}
\newcommand{\Z}{\mathbb{Z}}

\newenvironment{theorem}[2][Theorem]{\begin{trivlist}
\item[\hskip \labelsep {\bfseries #1}\hskip \labelsep {\bfseries #2.}]}{\end{trivlist}}
\newenvironment{lemma}[2][Lemma]{\begin{trivlist}
\item[\hskip \labelsep {\bfseries #1}\hskip \labelsep {\bfseries #2.}]}{\end{trivlist}}
\newenvironment{exercise}[2][Exercise]{\begin{trivlist}
\item[\hskip \labelsep {\bfseries #1}\hskip \labelsep {\bfseries #2.}]}{\end{trivlist}}
\newenvironment{reflection}[2][Reflection]{\begin{trivlist}
\item[\hskip \labelsep {\bfseries #1}\hskip \labelsep {\bfseries #2.}]}{\end{trivlist}}
\newenvironment{proposition}[2][Proposition]{\begin{trivlist}
\item[\hskip \labelsep {\bfseries #1}\hskip \labelsep {\bfseries #2.}]}{\end{trivlist}}
\newenvironment{corollary}[2][Corollary]{\begin{trivlist}
\item[\hskip \labelsep {\bfseries #1}\hskip \labelsep {\bfseries #2.}]}{\end{trivlist}}

\hyphenation{op-tical net-works semi-conduc-tor}

%% For our reference part
%\makeatletter
%\def\bstctlcite{\@ifnextchar[{\@bstctlcite}{\@bstctlcite[@auxout]}}
%\def\bstctlcite[#1]#2{\@bsphack\@for\@citeb:=#2\do{%
%\edef\@citeb{\expandafter\@firstofone\@citeb}%
%\if@filesw\immediate\write\csname #1\endcsname{\string\citation{\@citeb}}\fi}
%\@esphack}
%\makeatother

\begin{document}
%\bstctlcite{IEEEexample:BSTcontrol}

% --------------------------------------------------------------
%                         Start here
% --------------------------------------------------------------

\title{MIALab Project HS2020 - \\ Image normalization has an important influence on the segmentation}

\author{\IEEEauthorblockN{Thomas Buchegger} \IEEEauthorblockA{\textit{University of Bern,}}
\and \IEEEauthorblockN{Carolina Duran} \IEEEauthorblockA{\textit{University of Bern,}}
\and \IEEEauthorblockN{Stefan Weber,} \IEEEauthorblockA{\textit{University of Bern}}
\thanks{T. Buchegger, C. Duran and S. Weber are students from the University of Berne in Switzerland.}%
\thanks{E-mail of T. Buchegger: thomas.buchegger@students.unibe.ch}%
\thanks{E-mail of C. Duran: carolina.duran@students.unibe.ch}%
\thanks{E-mail of S. Weber: stefan.weber1@students.unibe.ch}}

\markboth{MIALab Project HS2020, January~2021}%
{Buchegger, Duran and Weber: Image normalization has an important influence on the segmentation}
\maketitle
%\pagestyle{plain}
\newpage


% --------------------------------------------------------------
% Abstract
% --------------------------------------------------------------

\begin{abstract}
Here is the abstract.


\end{abstract}
\hfill January 03, 2021



% --------------------------------------------------------------
% Introduction
% --------------------------------------------------------------

\section{Introduction}

	This is a very important citation test. \cite{Roohani2018}	
	Postprocesses of clinical diagnosis images for treatment planning often include manual segmentation of brain regions. 
	To segment and label the brain structures in a large dataset is complicated and time-consuming by manual operator-guided segmentation. 
	Furthermore, it is affected by user variability and prone to limiting the standardisation. 
	This paper will depend upon automated segmentation that segments and labels five different brain regions. \\


	To conclude we hypothesis that normalization has an important influence in the segmentation process of brain regions. 
	We present in this paper the acquired results applying multiple normalization methods and comparing them for every segmented brain region. 
	Furthermore, we analyse the results and conclude our findings. \\


\textbf{Our hypothesis: "Image normalization has an important influence on the segmentation".
	Aims of the introduction:
	\begin{itemize}
	\item To demonstrate importance/impact need
	\item To demonstrate novelty
	\item To justify the hypothesis / aims / investigated technology
	\item To establish expectations/scope of the report
	\end{itemize} 
	Some questions which could be answered:
	Currently, z score normalization is implemented.
	\begin{itemize}
	\item Are there more powerful normalization methods?
	\item Is normalization really needed?
	\item Is the provided data already normalized in some way?
	\item Can you also unnormalize data to show a negative effect?
	\end{itemize}
	For the research:
	\begin{itemize}
	\item Understand the problem
	\item What have others already done / is there already a solution?
	\item Can I apply a solution to another problem to my problem?
	\item What lessons can I learn from others work?
	\item What deficiencies exist in others work?
	\end{itemize}
	\begin{enumerate}
	\item Demonstrate importance
		\begin{enumerate}
		\item Define the problem
		\item Explain the criticality, impact of the problem
		\end{enumerate}
	\item Demonstrate Novelty
		\begin{enumerate}
		\item Explain what is the state of the art / current practice
		\item Explain what preliminary / related work has been done towards solving the problem by you and others
		\item Explain what deficiencies / problems still exist (specifically the one that you will try to address)
		\end{enumerate}
	\item Present and Justify the Hypothesis / aim / objective
		\begin{enumerate}
		\item State Hypothesis /aim/ objective
		\item Describe evidence supporting hypothesis
		\end{enumerate}
	\item Establish expectations of the report
		\begin{enumerate}
		\item Describe scope of the presented work
		\item Present a summary of remainder of the report\\
		\end{enumerate}
	\end{enumerate} 
}


% --------------------------------------------------------------
% Materials & Methods
% --------------------------------------------------------------

\section{Materials and Methods}
\textbf{MIA pipeline in general (short overview ), your experiment in detail}
\subsection{Medical Backgrund}
	The five brain regions segmented in this paper are the white and grey matter, hippocampus, amygdala and the thalamus. All regions are visible in figure \ref{fig:e1}.

	% --------------------------------------------------------------
	\begin{figure}[h]
		\centering
		%omit extension of file. pdflatex will convert to pdf automatically.
		\includegraphics[width=0.5\textwidth]{T1native_all_regions_labelled.png}
		\caption{All brain regions}
		\label{fig:figure1}
	\end{figure}
	% --------------------------------------------------------------

\subsection{Data}
	The used data were 30 unrelated healthy subjects from the Human Connectome Project data set. Overall, the dataset consisted of 30 MRI patient images,
	out of which 20 were used for training and 10 for testing the model. From each patient a 3 tesla T1- and a T2-weighted image (T1w and T2w image) was available. Each MRI file
	was of the size 118x118x217 pixels. All images are skull defaced for anonymization.
	Furthermore, the corresponding brain masks and the ground truth segmentation exist for all subjects. An atlas in MNI152 space is also available.

\subsection{Registration}
	To register the T1w and T2w image data an affine transformation was applied. The transformation was found by 
	an intersubject registration from the T1w image to the provided atlas. The corresponding transformations have already been determined
	and are not part of this work. 

\subsection{Preprocessing}
	As already stated the aim of this work is to determine the influence of the preprocessing and especially of the normalization on the performance of the segmentation.
	Therefore in total six different normalizations were applied to the image data. And the results were compared among all normalizations and no normalization.
	As an additional preprocessing step skull stripping was performed before normalization on all runs with the brain mask provided. To provide some context all used normalization techniques are briefly explained in this 
	section. The used normalizations already established and for additional information can be found. 

	The following normalization techniques were used:
		
	\subsubsection{ZScore}
		Normalize an image by subtracting the mean intensity and dividing by the standard deviation of the input image intensity.
		This transforms the image data into an intensity distribution with a mean of 0 and a standard deviation of 1.
			\begin{equation}\label{ZScore}
				I_{New} = \frac{I - \mu}{\sigma}
			\end{equation}
		
			\subsubsection{MinMax}
		Normalize an image by subtracting the minimal intensity value and dividing by the difference of the maximal and minimal intensities.
		This scales the intensities in a range from 0 to 1. 
			\begin{equation}
				I_{New} = \frac{I - I_{min}}{I_{max} - I_{min}}
			\end{equation}
		
		\subsubsection{Whitestripe}
		In the whitestripe normalization also formula (\ref{ZScore}) is used. In contrast to the Z-Score normalization $\mu$ and $\sigma$ obtained from the intensity
		values of normal appearing white matter (NAWM).  After smoothing the histogram NAWM is found by selecting the highest intensity peak of the T1-w image.
		$\mu$ corresponds to this peak. Around the obtained $\mu$ a subset of 10 \% from all pixel values is taken and the standard deviation $\sigma$ is calculated from this
		subset. This subset is called the whitestripe. By applying formula (\ref{ZScore}) with the found $\mu$ and  $\sigma$ the peak of the white matter is shifted to 0 and the intensities are scaled with $\sigma$.
		
		\subsubsection{Fuzzy-C Means} In this normalization a mask of the white matter is crated by using the Fuzzy C-means algorithm. 
		The found white matter tissue mask is used to calculate the mean $\mu$ of all intensities which correspond to the white matter. Then all image intensities are scaled by $\mu$ and shifted to a constant target value $c$
			\begin{equation}\label{FCM}
				I_{New} = \frac{c \cdot I}{\mu}
			\end{equation}

		\subsubsection{Gaussian Mixture Model}
		The Guassian Mixture Model normalization fits three Gaussian distributions to the skull stripped image intensities. The mean $\mu$ of Gaussian distribution which corresponds
		to the white matter is then used in a to normalize the image with the same formula (\ref{FCM}) as already used in the Fuzzy-C means normalization. $c$ is again the target value where the mean $\mu$ is shifted to. The mean $\mu$ corresponding to the white matter in a T1w image
		is the distribution with the highest intensities. In  a Tw2 image the white matter mean $\mu$ is the one with the lowest intensity values. 
		
		\subsubsection{Histogram Matching}
		Histogram matching manipulates the histogram of the input image so that the histogram of the output image matches with of a given reference image. As reference image the skull stripped T1w and T2w images of a subject were used. 
		This is done by mapping the cumulative distribution function of the input image to the reference image. 
		
		

\subsection{ Feature Extraction \& Classifier}
	The classifier used was a Random Forest classifier with which different parameters with all used normalization techniques were combined. Which parameters
	lead to which results can be found in the results section.

\subsection{Postprocessesing}

\subsection{Evaluation}
	For this project, the used normalization techniques were evaluated on the T1- and T2-weighted MRI data sets. 
	For this evaluation the Dice Similarity Coefficient (DSC) as well as the Hausdorff distance was chosen.
	DSC returns as a quality metric how much two regions overlap and is, therefore, a useful metric for segmentation.
	The dice coefficient of two sets, as explained in figure \ref{fig:e2}, is a measure of their intersection scaled by their size. The result is in the range from 0 to 1, where 1 is a perfect segmentation. 
	
	% --------------------------------------------------------------
	\begin{figure}[h]
		\centering
		%omit extension of file. pdflatex will convert to pdf automatically.
		\includegraphics[width=0.4\textwidth]{diceGraphics}
		\caption{Graphical representation of the Dice Similarity Coefficien. SEG stands for the achieved segmentation whereas GT means ground truth.}
		\label{fig:e2}
	\end{figure}
	% --------------------------------------------------------------

	The Hausdorff distance measures how far two subsets of a metric space are from each other. So two sets are close if every point of either set is close 
	to some point of the other set. Then the Hausdorff distance, explained in figure \ref{fig:e3}, is the longest distance from one of the two sets to the other set.

	% ----------------------------------------------------------zz
	\begin{figure}[h]
		\centering
		%omit extension of file. pdflatex will convert to pdf automatically.
<<<<<<< HEAD
		\includegraphics[width=0.3\textwidth]{haussdorfGraphics.png}
		\caption{Components of the calculation of the Hausdorff distance between the green line X and the blue line Y.}
		\label{fig:figure3}
=======
		\includegraphics[width=0.4\textwidth]{haussdorfGraphics}
		\caption{Components of the calculation of the Hausdorff distance between the green line X and the blue line Y.}
		\label{fig:e3}
>>>>>>> f368b2b55c36fdd80c03e7e927c60bab5ef27044
	\end{figure}
	% --------------------------------------------------------------


% --------------------------------------------------------------
% Results
% --------------------------------------------------------------

\section{Results}
\textbf{In depth analysis of experiment related results}

\subsection{Data}
\subsection{Model}
\subsection{Evaluation}

% --------------------------------------------------------------
% Discussion
% --------------------------------------------------------------

\section{Discussion}
\textbf{Aims of the Discussion part:
\begin{itemize}
\item Highlight importance of your work (highlight novelty /impact etc
\item To interpret your results in relation to your original problem
\item To put your work into the context of existing work
\item To present any limitations of the presented work
\item To make future recommendations
\item To provide a conclusion of the work
\end{itemize}
\begin{enumerate}
\item Importance of the work
	\begin{enumerate}
	\item Summarise your results
	\item Reiterate the importance of the work (novelty , impact etc)
	\end{enumerate}
\item Interpretation of results
	\begin{enumerate}
	\item Interpret your results focussing on the problem described in the introduction. What do the results mean for the described problem?
	\item Explain any unusual/important findings (be careful if not your original investigative subject)
	\end{enumerate}
\item Provide context
	\begin{enumerate}
	\item Describe your results in relation to others and try to explain any discrepancies
	\item Emphasize how your results support or refute your hypotheses current thinking in the field. Were results as expected? If not why and what does this mean?
	\end{enumerate}
\item Limitations of your work
	\begin{enumerate}
	\item Describe any limitations /deficiencies of your work and what impact they have on the findings
	\item Suggest possible future solutions
	\end{enumerate}
\end{enumerate} 
}

% --------------------------------------------------------------
% Conclusion
% --------------------------------------------------------------

\section{Conclusion}
\textbf{	
	\begin{itemize}
	\item Summarise your findings and relate your findings back to your hypothesis / aim / objective and to your problem.
	\item Based on your findings, suggest next steps towards solving your problem
	\end{itemize}
}
\pagebreak

% --------------------------------------------------------------
% References
% --------------------------------------------------------------

\bibliographystyle{ieeetr}
\bibliography{MIALab} 

\end{document}
